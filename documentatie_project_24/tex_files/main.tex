%%%%%%%%%%%%%%%%%%%%%%%%%%%%%%%%%%%%%%%%%
% The Legrand Orange Book
% LaTeX Template
% Version 2.3 (8/8/17)
%
% This template has been downloaded from:
% http://www.LaTeXTemplates.com
%
% Original author:
% Mathias Legrand (legrand.mathias@gmail.com) with modifications by:
% Vel (vel@latextemplates.com)
%
% License:
% CC BY-NC-SA 3.0 (http://creativecommons.org/licenses/by-nc-sa/3.0/)
%
% Compiling this template:
% This template uses biber for its bibliography and makeindex for its index.
% When you first open the template, compile it from the command line with the 
% commands below to make sure your LaTeX distribution is configured correctly:
%
% 1) pdflatex main
% 2) makeindex main.idx -s StyleInd.ist
% 3) biber main
% 4) pdflatex main x 2
%
% After this, when you wish to update the bibliography/index use the appropriate
% command above and make sure to compile with pdflatex several times 
% afterwards to propagate your changes to the document.
%
% This template also uses a number of packages which may need to be
% updated to the newest versions for the template to compile. It is strongly
% recommended you update your LaTeX distribution if you have any
% compilation errors.
%
% Important note:
% Chapter heading images should have a 2:1 width:height ratio,
% e.g. 920px width and 460px height.
%
%%%%%%%%%%%%%%%%%%%%%%%%%%%%%%%%%%%%%%%%%

%----------------------------------------------------------------------------------------
%	PACKAGES AND OTHER DOCUMENT CONFIGURATIONS
%----------------------------------------------------------------------------------------

\documentclass[11pt,fleqn,oneside,openany]{book} % Default font size and left-justified equations

%----------------------------------------------------------------------------------------

\input{structure} % Insert the commands.tex file which contains the majority of the structure behind the template

\newcommand{\titleText}{Leuke titel voor project} % to set the title
\newcommand{\subtitleText}{Documentatie verslag van project 2.4} % to set the subtitle
\newcommand{\authorText}{Dagmar Hoogendorp \\ Jelmer Wessels \\ Stèphan Hilbers \\ Roy van Kampen \\ Gijs Entius \\}

\begin{document}

%----------------------------------------------------------------------------------------
%	TITLE PAGE
%----------------------------------------------------------------------------------------

\begingroup
\thispagestyle{empty}
\noindent\makebox[\textwidth]{\includegraphics[width=\textwidth]{hanzehogeschool_image}}\\
\begin{tikzpicture}[remember picture,overlay]
% \node[inner sep=0pt] (background) at (current page.center) {\includegraphics[width=\paperwidth]{background_hanze}};
\draw (current page.center) node [fill=ocre!30!white,fill opacity=0.6,text opacity=1,inner sep=1cm]{\Huge\centering\bfseries\sffamily\parbox[c][][t]{\paperwidth}{\centering \titleText \\[15pt] % Book title
{\Large \subtitleText}\\[20pt] % Subtitle
{\huge \authorText}}}; % Author name
\end{tikzpicture}
\vfill
\endgroup

%----------------------------------------------------------------------------------------
%	COPYRIGHT PAGE
%----------------------------------------------------------------------------------------

\newpage
\thispagestyle{empty}

\noindent\makebox[\textwidth]{\includegraphics[width=\textwidth]{hanzehogeschool_image}}\\

\begin{tabbing}
    \textbf{Titel:} \hspace{2cm} \= \titleText \\
    \textbf{Ondertitel:} \> \subtitleText \\
    \textbf{Opleiding:} \> HBO-ICT, software engineering,leerjaar 2 \\
    \textbf{Auteurs:} \> Dagmar Hoogendorp (000000) \\
    \> Jelmer Wessels (359760) \\
    \> Roy van Kampen (000000) \\
    \> Gijs Entius (357665) \\
    \> Stèphan Hilberts (000000) \\
    \textbf{Plaats \& Datum:} \> Groningen, \today 
\end{tabbing}

% \begin{center}
%     \textbf{Titel:}\tab \titleText \par
%     \textbf{Ondertitel:}\tab \subtitleText \par
%     \textbf{Opleiding:}\tab HBO-ICT, software engineering,leerjaar 2 \par
%     \textbf{Autheur:}\tab \authorText \space (357665) \par
%     \textbf{Plaats \& Datum:}\tab Groningen, \today 
% \end{center}

\vfill
% ~\vfill % gebruik als er niks voor komt maar er wel vfill nodig is

\begin{center}
Copyright \copyright\ 2018 \\ Dagmar Hoogendorp, Jelmer Wessels, Stéphan Hilbers, Roy van Kampen \& Gijs Entius % Copyright notice
\end{center}

\begin{center}
    Dit rapport is geschreven onder verantwoordelijkheid van de Hanzehogeschool Groningen.
    Het copyright berust bij de auteur(s). Zowel de Hanzehogeschool Groningen als de auteur(s) verklaren, dat
    zij eventuele gegevens van derden die voor dit rapport zijn gebruikt en die door deze
    derden als vertrouwelijk zijn aangemerkt, als zodanig zullen behandelen.\\ % License information
\end{center}

%----------------------------------------------------------------------------------------
%	TABLE OF CONTENTS
%----------------------------------------------------------------------------------------

\usechapterimagefalse % If you don't want to include a chapter image, use this to toggle images off - it can be enabled later with \usechapterimagetrue

\chapterimage{chapter_head_1.pdf} % Table of contents heading image

\pagestyle{empty} % No headers

\tableofcontents % Print the table of contents itself

% \cleardoublepage % Forces the first chapter to start on an odd page so it's on the right

\pagestyle{fancy} % Print headers again

%----------------------------------------------------------------------------------------
%   Chapters
%----------------------------------------------------------------------------------------

\chapter{Inleiding}

\chapter{Projectopzet}

\chapter{Case}
\section{Weekplannig}
\subsection{Sprintverdeling}
Sprint 1: Dagmar \\
Sprint 2: Jelmer \\
Sprint 3: Roy \\
Sprint 4: Gijs \\
Sprint 5: Stèphan \\

\section{Concept}
Vrienden toevoegen
Bedrijfs integration

Pagina voor inloggen, webpagina (profiel), homepagina / newsfeed

\section{Gameification}
Uitdagingen halen -> punten scoren
Uitdagingen door bedrijven, users en developers
User uitdgingen -> punten toekennen op basis van categoriën en evntueel moeilijkheidsgraad
Uitdagingen delen met vrienden via newsfeed, maar ook via twitter
Badges halen

\section{Froud}
Er kan niks gewonnen worden
Vertrouwen

\section{Integratie met andere websites}
Google account
Integratie met Facebook

\section{Vinden van gebruikers}
Facebook vrienden
Andere gebruikers

\section{Uitgeschreven case}
Case <naam>
Concept
-----------------------------
Winstmodel
------------------------------
Gamification
Het game component van de applicatie bestaat uit challenges. Voorbeelden van deze challenges zijn een vegetarisch recept maken of het gebruiken van een droogmolen in plaats van een droger. Met deze challenges kunnen punten worden verdiend. Het behalen van een challenge kan ook gedeeld worden met vrienden op het platform of een geïntegreerd platform, zoals twitter of facebook.

Met de punten die gescoord worden kunnen badges behaald worden, welke weergeven worden op het profiel van de gebruiker. Er zijn verder geen beloningen voor het halen van challenges.

De challenges worden door drie partijen gemaakt. Allereerst door de developers van de applicatie. Daarnaast door bedrijven in samenwerking met de developers, en ten slotte door de gebruikers zelf. Bedrijven mogen hun groene, milieuvriendelijke of duurzame producten promoten door ze aan te bevelen binnen een challenge voor een kleine vergoeding.

Het aantal punten dat aan een challenge wordt toegekend wordt berekend aan de hand van categorieën en moeilijkheidsgraad. De challenges van gebruikers kunnen ingestuurd worden via de applicatie of automatisch ingevoerd worden door middel van een systeem met verschillende opties dat een acceptabel puntenaantal toekent.
Fraude
Gebruikers kunnen zelf aangeven wanneer ze een challenge gehaald hebben. Het is moeilijk te controleren of de gebruikers eerlijk zijn. Daarom is ervoor gekozen om geen beloningen uit te geven behalve de badges. Het is aan de gebruiker zelf om eerlijk te zijn. Bij het delen van challenges zal dan ook een foto toegevoegd moeten worden waar dit relevant is.
Het is uiteindelijk aan de gebruikers zelf om eerlijk te zijn tegenover zichzelf en zijn of haar vrienden en heeft alleen zichzelf ermee als challenges niet eerlijk worden volbracht.
Integratie met andere websites
-------------------------
Vinden van nieuwe gebruikers
Gebruikers kunnen vrienden die de applicatie al gebruiken vinden door middel van een zoekfunctie. Nieuwe gebruikers kunnen worden aangetrokken doordat challenges gedeeld kunnen worden op andere platformen en zo groeit de bekendheid van de applicatie. Daarnaast kan een gebruiker potentiële gebruikers uitnodigen om een account aan te maken.


\section{Eigenschappen}

%----------------------------------------------------------------------------------------
%	APPENDICES
%----------------------------------------------------------------------------------------

\chapter*{Bijlagen}


%----------------------------------------------------------------------------------------
%	Examples
%----------------------------------------------------------------------------------------

\part{Examples}

%----------------------------------------------------------------------------------------
%	CHAPTER 1
%----------------------------------------------------------------------------------------

\chapterimage{chapter_head_2.pdf} % Chapter heading image

\chapter{Text Chapter}

\section{Paragraphs of Text}\index{Paragraphs of Text}

\lipsum[1-7] % Dummy text

%------------------------------------------------

% \section{Citation}\index{Citation}

% This statement requires citation \cite{article_key}; this one is more specific \cite[162]{book_key}.

%------------------------------------------------

\section{Lists}\index{Lists}

Lists are useful to present information in a concise and/or ordered way\footnote{Footnote example...}.

\subsection{Numbered List}\index{Lists!Numbered List}

\begin{enumerate}
\item The first item
\item The second item
\item The third item
\end{enumerate}

\subsection{Bullet Points}\index{Lists!Bullet Points}

\begin{itemize}
\item The first item
\item The second item
\item The third item
\end{itemize}

\subsection{Descriptions and Definitions}\index{Lists!Descriptions and Definitions}

\begin{description}
\item[Name] Description
\item[Word] Definition
\item[Comment] Elaboration
\end{description}

%----------------------------------------------------------------------------------------
%	CHAPTER 2
%----------------------------------------------------------------------------------------

\chapter{In-text Elements}

\section{Theorems}\index{Theorems}

This is an example of theorems.

\subsection{Several equations}\index{Theorems!Several Equations}
This is a theorem consisting of several equations.

\begin{theorem}[Name of the theorem]
In $E=\mathbb{R}^n$ all norms are equivalent. It has the properties:
\begin{align}
& \big| ||\mathbf{x}|| - ||\mathbf{y}|| \big|\leq || \mathbf{x}- \mathbf{y}||\\
&  ||\sum_{i=1}^n\mathbf{x}_i||\leq \sum_{i=1}^n||\mathbf{x}_i||\quad\text{where $n$ is a finite integer}
\end{align}
\end{theorem}

\subsection{Single Line}\index{Theorems!Single Line}
This is a theorem consisting of just one line.

\begin{theorem}
A set $\mathcal{D}(G)$ in dense in $L^2(G)$, $|\cdot|_0$. 
\end{theorem}

%------------------------------------------------

\section{Definitions}\index{Definitions}

This is an example of a definition. A definition could be mathematical or it could define a concept.

\begin{definition}[Definition name]
Given a vector space $E$, a norm on $E$ is an application, denoted $||\cdot||$, $E$ in $\mathbb{R}^+=[0,+\infty[$ such that:
\begin{align}
& ||\mathbf{x}||=0\ \Rightarrow\ \mathbf{x}=\mathbf{0}\\
& ||\lambda \mathbf{x}||=|\lambda|\cdot ||\mathbf{x}||\\
& ||\mathbf{x}+\mathbf{y}||\leq ||\mathbf{x}||+||\mathbf{y}||
\end{align}
\end{definition}

%------------------------------------------------

\section{Notations}\index{Notations}

\begin{notation}
Given an open subset $G$ of $\mathbb{R}^n$, the set of functions $\varphi$ are:
\begin{enumerate}
\item Bounded support $G$;
\item Infinitely differentiable;
\end{enumerate}
a vector space is denoted by $\mathcal{D}(G)$. 
\end{notation}

%------------------------------------------------

\section{Remarks}\index{Remarks}

This is an example of a remark.

\begin{remark}
The concepts presented here are now in conventional employment in mathematics. Vector spaces are taken over the field $\mathbb{K}=\mathbb{R}$, however, established properties are easily extended to $\mathbb{K}=\mathbb{C}$.
\end{remark}

%------------------------------------------------

\section{Corollaries}\index{Corollaries}

This is an example of a corollary.

\begin{corollary}[Corollary name]
The concepts presented here are now in conventional employment in mathematics. Vector spaces are taken over the field $\mathbb{K}=\mathbb{R}$, however, established properties are easily extended to $\mathbb{K}=\mathbb{C}$.
\end{corollary}

%------------------------------------------------

\section{Propositions}\index{Propositions}

This is an example of propositions.

\subsection{Several equations}\index{Propositions!Several Equations}

\begin{proposition}[Proposition name]
It has the properties:
\begin{align}
& \big| ||\mathbf{x}|| - ||\mathbf{y}|| \big|\leq || \mathbf{x}- \mathbf{y}||\\
&  ||\sum_{i=1}^n\mathbf{x}_i||\leq \sum_{i=1}^n||\mathbf{x}_i||\quad\text{where $n$ is a finite integer}
\end{align}
\end{proposition}

\subsection{Single Line}\index{Propositions!Single Line}

\begin{proposition} 
Let $f,g\in L^2(G)$; if $\forall \varphi\in\mathcal{D}(G)$, $(f,\varphi)_0=(g,\varphi)_0$ then $f = g$. 
\end{proposition}

%------------------------------------------------

\section{Examples}\index{Examples}

This is an example of examples.

\subsection{Equation and Text}\index{Examples!Equation and Text}

\begin{example}
Let $G=\{x\in\mathbb{R}^2:|x|<3\}$ and denoted by: $x^0=(1,1)$; consider the function:
\begin{equation}
f(x)=\left\{\begin{aligned} & \mathrm{e}^{|x|} & & \text{si $|x-x^0|\leq 1/2$}\\
& 0 & & \text{si $|x-x^0|> 1/2$}\end{aligned}\right.
\end{equation}
The function $f$ has bounded support, we can take $A=\{x\in\mathbb{R}^2:|x-x^0|\leq 1/2+\epsilon\}$ for all $\epsilon\in\intoo{0}{5/2-\sqrt{2}}$.
\end{example}

\subsection{Paragraph of Text}\index{Examples!Paragraph of Text}

\begin{example}[Example name]
\lipsum[2]
\end{example}

%------------------------------------------------

\section{Exercises}\index{Exercises}

This is an example of an exercise.

\begin{exercise}
This is a good place to ask a question to test learning progress or further cement ideas into students' minds.
\end{exercise}

%------------------------------------------------

\section{Problems}\index{Problems}

\begin{problem}
What is the average airspeed velocity of an unladen swallow?
\end{problem}

%------------------------------------------------

\section{Vocabulary}\index{Vocabulary}

Define a word to improve a students' vocabulary.

\begin{vocabulary}[Word]
Definition of word.
\end{vocabulary}

%----------------------------------------------------------------------------------------
%	CHAPTER 3
%----------------------------------------------------------------------------------------

\chapterimage{chapter_head_1.pdf} % Chapter heading image

\chapter{Presenting Information}

\section{Table}\index{Table}

\begin{table}[h]
\centering
\begin{tabular}{l l l}
\toprule
\textbf{Treatments} & \textbf{Response 1} & \textbf{Response 2}\\
\midrule
Treatment 1 & 0.0003262 & 0.562 \\
Treatment 2 & 0.0015681 & 0.910 \\
Treatment 3 & 0.0009271 & 0.296 \\
\bottomrule
\end{tabular}
\caption{Table caption}
\end{table}

%------------------------------------------------

\section{Figure}\index{Figure}

\begin{figure}[h]
\centering\includegraphics[scale=0.5]{placeholder}
\caption{Figure caption}
\end{figure}

%----------------------------------------------------------------------------------------
%	BIBLIOGRAPHY
%----------------------------------------------------------------------------------------

% \chapter*{Bibliography}
% \addcontentsline{toc}{chapter}{\textcolor{ocre}{Bibliography}}

%------------------------------------------------

% \section*{Articles}
% \addcontentsline{toc}{section}{Articles}
% \printbibliography[heading=bibempty,type=article]

% %------------------------------------------------

% \section*{Books}
% \addcontentsline{toc}{section}{Books}
% \printbibliography[heading=bibempty,type=book]

\end{document}
